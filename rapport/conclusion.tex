\section{Conclusion et pistes de recherche future}
Notre équipe a développer un gestionnaire de classificateur ainsi que l'ensemble des classificateurs utiliser par celui-ci. Il permet de faire une recherche d'hyper-paramètres et d'en sortir le meilleure modèle selon la performance de celui-ci. Il fait la classification des feuille présentes dans l'ensemble de données \emph{leaf-classification} que l'on retrouve sur le site web \textbf{kaggle}.

En rétrospective, la plus part des modèles ont des performances dans les 100\% de bonne classifications lors de la validation et majoritairement au-delà des 95\% en teste de validation. Les trois modèles les plus performant sont \emph{Random Forest}(98.48\%), \emph{LDA}(97.98\%) et \emph{KNN}(97.47\%).

En ce qui concerne le futur, les réseaux de neurones est une technique d'apprentissage qui a fait plus que ses preuves et elle est maintenant rendu universelle dans des problèmes de classifications. Il serait bien d'évaluer les performances de cette technique par rapport aux autres de ce projet pour faire un comparatif entre les anciennes méthodes et les nouvelles.