\section{Mise en contexte}
Ce projet a pour but de présenter différentes méthodes de classification dans le cadre d'un apprentissage supervisée. 

Ce projet vise également à mettre en application des méthodes de cross-validation et de recherche d'hyper-paramètres dans le but de déterminer les modèles de classification les plus performants. 
\section{Définition du projet}
Ce présent projet classifie un ensemble de feuille, composé d'information extrait de 1584 images de feuille, selon des caractéristiques physiques. Les caractéristique, soit la marge à échelle précise, la forme et la texture des feuilles de l'ensemble, sont des vecteurs 3x64.

Nous avons développer un gestionnaire qui choisi un classificateur spécifique ou bien l'ensemble de 7 classificateur. Ce gestionnaire permet, une fois le ou les classificateurs choisis, de faire une recherche d'hyper-paramètres, d'entrainer un ou plusieurs modèle et de générer les informations sur la performance du ou des modèles. 
    \label{sec:definition_projet}
    
    \pagebreak
    \begin{figure}[H]
        \centering
        %\includegraphics[width=15cm]{images/}
        \caption[Définition de la sortie du réseau]{Définition de la sortie du réseau\footnotemark}
        \label{fig:definition_sortie_reseau}
    \end{figure}